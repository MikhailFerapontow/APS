\subsection{Невытесняющий алгоритм планирования}

\noindent
\textbf{Цель теста}: Проверка работы невытесняющего алгоритма планирования.

\noindent
\textbf{Ожидаемый результат }: Задача низкого приоритета завершает своё выполнение без
прерываний, потом срабатывает задача высокого приоритета, после неё –- задача среднего
приоритета;

\noindent
\textbf{Результат}: 
\inputminted[linenos=false]{text}{results/test1.txt}

\subsection{Системные события}

\noindent
\textbf{Цель теста}: Проверка работы системных событий.

\noindent
\textbf{Ожидаемый результат}: Событие перевелось в состояние ожидания по системному событию пока задача, которая
поставит это событие не закончит свою работу.

\noindent
\textbf{Результат}: 
\inputminted[linenos=false]{text}{results/test2.txt}
\subsection{PIP-алгоритм}

\noindent
\textbf{Цель теста}: Проверка работы PIP-алгоритма

\noindent
\textbf{Ожидаемый результат}: Задача низкого приоритета должна успешно захватить и освободить
ресурс, предоставив задаче высокого приоритета возможность успешно захватить
его снова

\noindent
\textbf{Результат}: 
\inputminted[linenos=false]{text}{results/test3.txt}
\subsection{Полный функционал}

\noindent
\textbf{Цель теста}: Проверка работы всех функций вместе.

\noindent
\textbf{Результат}: 
\inputminted[linenos=false]{text}{results/test4.txt}