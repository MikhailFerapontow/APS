\subsection{Плоский планировщик}
Плоский планировщик (\textit{flat scheduler}) - это тип планировщика операционной системы,
который использует простейший алгоритм планирования, такой как "первым пришел - первым обслужен"
(\textit{First Come, First Served, FCFS}).
При таком подходе процессы выполняются в том порядке, в котором они поступили в очередь на выполнение, 
без учета их приоритетов или времени выполнения.

\subsection{Невытесняющий алгоритм планирования}
В данном варианте используется \textit{nonpreemptive} или невытесняющий алгоритм планирования.
Этот алгоритм основан на том, что активному потоку позволяется выполняться, пока он сам,
по собственной инициативе, не отдаст управление операционной системе для того, чтобы та
выбрала из очереди другой готовый к выполнению поток.

\subsection{Протокол наследования приоритетов}
Протокол наследования приоритетов (\textit{PIP}) используется для совместного использования 
критических ресурсов между различными задачами без возникновения неограниченных инверсий приоритета
\footnote[1]{Инверсия приоритета - это ситуация, которая возникает при параллельном выполнении нескольких 
задач с разными приоритетами, когда задача с более высоким приоритетом блокируется задачей с более низким
 приоритетом из-за необходимости доступа к общему ресурсу}.

Когда несколько задач ждут доступа к одному и тому же критическому ресурсу, которая
в данный момент имеет доступ к этому ресурсу, присваивается самый высокий приоритет среди всех
задач, ждущих этот ресурс. Если задача с более низким приоритетом, но имеющая доступ к критическому
ресурсу, находится в ожидании, то ее приоритет увеличивается до приоритета задачи, которая в данный
момент имеет доступ к этому ресурсу. Это позволяет нижестоящей задаче использовать критический ресурс
без прерывания выполнения и избежать неограниченных инверсий приоритета.

