\documentclass[12pt, a4paper]{article}

\usepackage[T2A]{fontenc}
\usepackage[utf8]{inputenc}
\usepackage[english, russian]{babel}
\usepackage{amsmath, amsfonts, amsthm}
\usepackage{listings}
\usepackage{enumerate}
\usepackage{float}
\usepackage{graphicx}
\usepackage{nameref}
\usepackage{hyperref}
\usepackage{tabularx}
\usepackage{indentfirst}
\usepackage{booktabs}
\usepackage{subcaption}
\usepackage{parskip}

\usepackage[newfloat]{minted}

\setminted{
  frame=lines,
  framesep=2mm,
  baselinestretch=1.2,
  fontsize=\footnotesize,
  linenos
}

\usepackage[top=3cm,bottom=3cm,left=2.5cm,right=2.5cm]{geometry}
\pagenumbering{arabic}

\begin{document}

\begin{titlepage}	% начало титульной страницы

	\begin{center}		% выравнивание по центру

		\large Санкт-Петербургский политехнический университет Петра Великого\\
		\large Институт компьютерных наук и технологий\\
		\large Высшая школа программной инженерии \\[6cm]
		% название института, затем отступ 6см

    \huge Курсовая работа \\% название работы, затем отступ 0,5см
		\large Модель операционной системы реального времени \\ [0.5cm]
		\large по дисциплине\\[0.1cm]
		\large <<Архитектура программных систем>>\\[5cm]

	\end{center}

		\noindent\large Выполнил: \hfill \large Савчук А.А.\\
		\noindent\large Группа: \hfill \large гр. 3530904/00104\\

		\noindent\large Проверил: \hfill \large Коликова Т. В.

	\vfill % заполнить всё доступное ниже пространство

	\begin{center}
	\large Санкт-Петербург\\
	\large \the\year % вывести дату
	\end{center} % закончить выравнивание по центру

\end{titlepage} % конец титульной страницы

\vfill % заполнить всё доступное ниже пространство


\tableofcontents
\newpage

\section{Постановка задачи}

\textbf{Индвидуальное задание №4} \\
Необходимо реализовать модель операционной системы реального времени обладающей следующими свойствами:

\noindent
\textbf{Тип планировщика:} плоский

\noindent
\textbf{Алгоритм планирования:} nonpreemptive, RMA

\noindent
\textbf{Управление ресурсами:} PIP

\noindent
\textbf{Управление событиями:} системные события

\noindent
\textbf{Обработка прерываний:} нет

\noindent
\textbf{Максимальное количество задач:} 32

\noindent
\textbf{Максимальное количество ресурсов:} 16

\noindent
\textbf{Максимальное количество событий:} 16

\noindent
Кроме того, в задание входит написание тестов проверяющих соответствие проекта этим свойствам.
\newpage

\section{Пояснения к варианту}
\subsection{Плоский планировщик}
Плоский планировщик (\textit{flat scheduler}) - это тип планировщика операционной системы,
который использует простейший алгоритм планирования, такой как "первым пришел - первым обслужен"
(\textit{First Come, First Served, FCFS}).
При таком подходе процессы выполняются в том порядке, в котором они поступили в очередь на выполнение, 
без учета их приоритетов или времени выполнения.

\subsection{Невытесняющий алгоритм планирования}
В данном варианте используется \textit{nonpreemptive} или невытесняющий алгоритм планирования.
Этот алгоритм основан на том, что активному потоку позволяется выполняться, пока он сам,
по собственной инициативе, не отдаст управление операционной системе для того, чтобы та
выбрала из очереди другой готовый к выполнению поток.

\subsection{Протокол наследования приоритетов}
Протокол наследования приоритетов (\textit{PIP}) используется для совместного использования 
критических ресурсов между различными задачами без возникновения неограниченных инверсий приоритета
\footnote[1]{Инверсия приоритета - это ситуация, которая возникает при параллельном выполнении нескольких 
задач с разными приоритетами, когда задача с более высоким приоритетом блокируется задачей с более низким
 приоритетом из-за необходимости доступа к общему ресурсу}.

Когда несколько задач ждут доступа к одному и тому же критическому ресурсу, которая
в данный момент имеет доступ к этому ресурсу, присваивается самый высокий приоритет среди всех
задач, ждущих этот ресурс. Если задача с более низким приоритетом, но имеющая доступ к критическому
ресурсу, находится в ожидании, то ее приоритет увеличивается до приоритета задачи, которая в данный
момент имеет доступ к этому ресурсу. Это позволяет нижестоящей задаче использовать критический ресурс
без прерывания выполнения и избежать неограниченных инверсий приоритета.


\newpage

\section{Описание структуры проекта}
\subsection{Управление ОС}

\begin{description}
  \item[StartOS(entry, priority, name)] -- Запуск ОС, инициализация основных элементов системы, активация начальной задачи.
  \item[ShutdownOS()] -- Завершение работы ОС. 
\end{description}

\subsection{Управление задачами}

\begin{description}
  \item[ActivateTask(entry, priority, name)] -- Инициализация задачи в системе.
  \item[TermitateTask()] -- Завершение задачи.
  \item[Schedule(task, mask)] -- Постановка задачи в очередь.
  \item[Dispatch()] -- Диспетчеризация задач, постановка на выполнение.
\end{description}

\subsection{Управление ресурсами}

\begin{description}
  \item[InitRes(name)] -- Инициализация ресурса в системе.
  \item[PIP\_GetRes(res)] -- Захват ресурса выполняющейся задачей.
  \item[PIP\_ReleaseRes(res)] -- Освобождение ресурса.
\end{description}

\subsection{Управление событиями}
\begin{description}
  \item[SetSysEvent(mask)] -- Установка системного события.
  \item[GetSysEvent(mask)] -- Возвращает текущее состояние системной маски установленных событий.
  \item[WaitSysEvent(mask)] -- Задача переводится в состояние ожидания.
\end{description}
\newpage

\section{Тестирование}
\subsection{Невытесняющий алгоритм планирования}

\noindent
\textbf{Цель теста}: Проверка работы невытесняющего алгоритма планирования.

\noindent
\textbf{Ожидаемый результат }: Задача низкого приоритета завершает своё выполнение без
прерываний, потом срабатывает задача высокого приоритета, после неё –- задача среднего
приоритета;

\noindent
\textbf{Результат}: 
\inputminted[linenos=false]{text}{results/test1.txt}

\subsection{Системные события}

\noindent
\textbf{Цель теста}: Проверка работы системных событий.

\noindent
\textbf{Ожидаемый результат}: Событие перевелось в состояние ожидания по системному событию пока задача, которая
поставит это событие не закончит свою работу.

\noindent
\textbf{Результат}: 
\inputminted[linenos=false]{text}{results/test2.txt}
\subsection{PIP-алгоритм}

\noindent
\textbf{Цель теста}: Проверка работы PIP-алгоритма

\noindent
\textbf{Ожидаемый результат}: Задача низкого приоритета должна успешно захватить и освободить
ресурс, предоставив задаче высокого приоритета возможность успешно захватить
его снова

\noindent
\textbf{Результат}: 
\inputminted[linenos=false]{text}{results/test3.txt}
\subsection{Полный функционал}

\noindent
\textbf{Цель теста}: Проверка работы всех функций вместе.

\noindent
\textbf{Результат}: 
\inputminted[linenos=false]{text}{results/test4.txt}
\newpage

\section*{Код Программы}
\addcontentsline{toc}{section}{\protect\numberline{}Код программы}%
\inputminted[label=global.cpp]{c++}{../global.cpp}
\inputminted[label=defs.h]{c++}{../defs.h}
\inputminted[label=rtos\_api.h]{c++}{../rtos_api.h}
\inputminted[label=sys.h]{c++}{../sys.h}
\inputminted[label=events.cpp]{c++}{../events.cpp}
\inputminted[label=resource.cpp]{c++}{../resource.cpp}
\inputminted[label=task.cpp]{c++}{../task.cpp}
\inputminted[label=test.cpp]{c++}{../test.cpp}

\end{document}